\documentclass[tikz]{standalone}

\usepackage{tkz-euclide}
\usetkzobj{all}
\usepackage{tkz-base}


\makeatletter
\def\tkzExchangeCoord(#1,#2){% 
	\tkz@@extractxy{#1}%
	\pgf@xa=\pgf@x\relax
	\tkz@@extractxy{#2}%
	\pgf@yb=\pgf@y\relax 
	\path[coordinate](\pgf@xa,\pgf@yb) coordinate (tkzPointResult);}
\makeatother


\begin{document}
	\begin{tikzpicture}[scale=1]
		% Punkte definieren
		\tkzDefPoint(0,0){A}
		\tkzDefShiftPointCoord[A](0:8){B}
		\tkzDefShiftPointCoord[A](45:9){C}
		
		% Zeichne das Dreick
		\tkzDrawPolygon(A,B,C)
		
		% Punkte zeichen und bezeichnen
		\tkzDrawPoints(A,B,C)
		\tkzLabelPoints[below](A,B)
		\tkzLabelPoints[above](C)
		
		% Dreiecksseiten bezeichnen
		\tkzLabelSegment[above left](A,C){$b$}
		\tkzLabelSegment[above right](B,C){$a$}
		\tkzLabelSegment[below](A,B){$c$}
		
		% Winkel markieren
		\tkzMarkAngle[arc=l,size=1.1cm](B,A,C)
		\tkzLabelAngle[pos = 0.9](C,A,B){$\alpha$}
		\tkzMarkAngle[arc=l,size=0.8cm](C,B,A)
		\tkzLabelAngle[pos = 0.6](C,B,A){$\beta$}
		\tkzMarkAngle[arc=l,size=0.9cm](A,C,B)
		\tkzLabelAngle[pos = 0.6](B,C,A){$\gamma$}
		
		% Gebe die Winkelgrößen aus
		%\tkzFindAngle(B,A,C)\tkzGetAngle{Walpha}
		%\tkzFindAngle(C,B,A)\tkzGetAngle{Wbeta}
		%\tkzFindAngle(A,C,B)\tkzGetAngle{Wgamma}
		%\tkzText(-1,-1){$ \alpha =  $ \Walpha}
		%\tkzText(-1,-2){$ \beta =  $ \Wbeta}
		%\tkzText(-1,-3){$ \gamma =  $ \Wgamma}
		
		
		% Winkelhalbierende bestimmen
		\tkzDefLine[bisector](B,A,C) \tkzGetPoint{wa}
		\tkzDefLine[bisector](C,B,A) \tkzGetPoint{wb}
		\tkzDefLine[bisector](A,C,B) \tkzGetPoint{wc}
		% Schnittpunkt (Innenkreismittelpunkt) bestimmen
		\tkzInterLL(A,wa)(B,wb) \tkzGetPoint{W}
		%%% Alternativ
		%\tkzInCenter(A,B,C)\tkzGetPoint{W}
		
		% Winkelhalbierende zeichnen
		%\tkzDrawLines(A,wa B,wb C,wc)
		% Umkreis zeichnen
		\tkzDefCircle[in](A,B,C)
		\tkzGetLength{rIN} %\tkzGetPoint{I}
		\tkzDrawCircle[R,blue](W,\rIN pt)
		
		% Mittelpunkte der Seiten definieren
		\tkzDefMidPoint(A,B)\tkzGetPoint{Mab}
		\tkzDefMidPoint(A,C)\tkzGetPoint{Mac}
		\tkzDefMidPoint(B,C)\tkzGetPoint{Mbc}
		
		% Mittelsenkrechte definieren
		\tkzDefLine[orthogonal=through Mab](A,B) \tkzGetPoint{mab}
		\tkzDefLine[orthogonal=through Mac](C,A) \tkzGetPoint{mac}
		\tkzDefLine[orthogonal=through Mbc](B,C) \tkzGetPoint{mbc}
		% Schnittpunkt (Umkreismittelpunk) bestimmen
		\tkzInterLL(Mab,mab)(Mac,mac) \tkzGetPoint{M}
		%%% Alternativ
		%\tkzCircumCenter(A,B,C)\tkzGetPoint{M}
		
		% Mittelsenkrechte zeichnen
		%\tkzDrawLines(Mab,mab Mac,mac Mbc,mbc)
		% Umkreis zeichnen
		\tkzDrawCircle(M,A)
		%\tkzDefCircle[circum](A,B,C)
		%\tkzGetLength{rCI} %\tkzGetPoint{K}
		%\tkzDrawCircle[R,red](M,\rCI pt)
		
		%Schwerpunkt definieren
		%\tkzDefBarycentricPoint(A=1,B=1,C=1)\tkzGetPoint{S}
		\tkzCentroid(A,B,C)\tkzGetPoint{S}
		
		% Höhen definieren
		\tkzDefLine[orthogonal=through A](C,B) \tkzGetPoint{ha}
		\tkzDefLine[orthogonal=through B](A,C) \tkzGetPoint{hb}
		\tkzDefLine[orthogonal=through C](B,A) \tkzGetPoint{hc}
		% Schnittpunkt bestimmen
		\tkzInterLL(A,ha)(B,hb) \tkzGetPoint{H}
		% Höhen zeichnen
		%\tkzDrawLines(A,ha B,hb C,hc)
		
		% besondere Punkte zeichnen
		\tkzDrawPoints(S,W,M,H)
		% besondere Punkte bezeichnen
		\tkzLabelPoints[below](S,M,H)
		\tkzLabelPoints[above](W)
	\end{tikzpicture}
\end{document}
